\documentclass[12pt,a4paper]{article}

\usepackage[utf8]{inputenc}
\usepackage[T1]{fontenc}
\usepackage[portuguese]{babel}
\usepackage{amsmath,amsfonts,amssymb}
\usepackage{graphicx}
\usepackage[a4paper,margin=2cm]{geometry}
\usepackage{float}
\usepackage{hyperref}
\usepackage{caption}
\usepackage{cite}
\usepackage{listings}
\usepackage{enumitem}
\usepackage{xcolor}

\lstset{
  language=Python,
  basicstyle=\ttfamily\small,
  numbers=left,
  numberstyle=\tiny,
  frame=single,
  breaklines=true,
  keywordstyle=\color{blue}\bfseries,
  stringstyle=\color{red},
  commentstyle=\color{green!60!black}\itshape,
  showstringspaces=false
}

\graphicspath{{./}{../}{../../}{../../plots/}{../../plots/img/}}

\title{
{\footnotesize Universidade Federal de Minas Gerais\\
Escola de Engenharia\\
ELE077 – Otimização Não Linear\\}
\vspace{5mm}
\textbf{Trabalho Computacional II : Otimização Não Linear}
}

\author{
Áquila Oliveira Souza — 2021019327 \and
Josoé Santos Queiroz — 2019026982
}

\date{}

\begin{document}

\maketitle



\section{Introdução}



\section{Problema 1: Treliça de Três Barras}

O objetivo do problema de otimização restrita é minimizar o gasto de material de uma treliça de três barras. Sendo que $x_1 = A_1$ é a área da seção transversal das barras laterais e $x_2 = A_2$ é a área da seção transversal da barra central.

A função objetivo representa o volume total (ou peso, proporcional ao comprimento) das barras:

\begin{equation}
    \min f(x_1, x_2) = (2\sqrt{2})x_1 + x_2
\end{equation}

As restrições impostas ao problema são baseadas nos limites de estresse e dimensões físicas:

1. Estresse máximo das barras 1 e 2:

\begin{equation}
    P \frac{x_2 + x_1 \sqrt{2}}{x_1^2 \sqrt{2} + 2x_1 x_2} \leq 20
\end{equation}

\begin{equation}
    P \frac{1}{x_1 + x_2 \sqrt{2}} \leq 20
\end{equation}

2. Estresse mínimo para a barra 3:

\begin{equation}
    -P \frac{x_2}{x_1^2 \sqrt{2} + 2x_1 x_2} \leq -5
\end{equation}

3. Limites de tamanho para as barras:

\begin{// filepath: c:\Users\User\OneDrive\Documentos\GitHub\nlo-final-assignment\relatorio_final.tex
// ...existing code...
\section{Introdução}



\section{Problema 1: Treliça de Três Barras}

O objetivo do problema de otimização restrita é minimizar o gasto de material de uma treliça de três barras. Sendo que $x_1 = A_1$ é a área da seção transversal das barras laterais e $x_2 = A_2$ é a área da seção transversal da barra central.

A função objetivo representa o volume total (ou peso, proporcional ao comprimento) das barras:

\begin{equation}
    \min f(x_1, x_2) = (2\sqrt{2})x_1 + x_2
\end{equation}

As restrições impostas ao problema são baseadas nos limites de estresse e dimensões físicas:

1. Estresse máximo das barras 1 e 2:

\begin{equation}
    P \frac{x_2 + x_1 \sqrt{2}}{x_1^2 \sqrt{2} + 2x_1 x_2} \leq 20
\end{equation}

\begin{equation}
    P \frac{1}{x_1 + x_2 \sqrt{2}} \leq 20
\end{equation}

2. Estresse mínimo para a barra 3:

\begin{equation}
    -P \frac{x_2}{x_1^2 \sqrt{2} + 2x_1 x_2} \leq -5
\end{equation}

3. Limites de tamanho para as barras:

\begin{equation}
    0.1 \leq x_1, x_2 \leq 5
\end{equation}

Onde a carga aplicada é $P = 10$.


\section{Problema 2: Despacho Econômico de Energia}
O problema consiste em determinar a distribuição ótima de geração entre as unidades geradoras disponíveis para atender à demanda de energia elétrica ao menor custo possível.

A função objetivo é minimizar o custo total de geração:

\begin{equation}
    \min f(P_1, P_2, P_3) = C_1(P_1) + C_2(P_2) + C_3(P_3)
\end{equation}

Onde as curvas de custo são dadas por:

\begin{align*}
    C_1(P_1) &= 0.15 P_1^2 + 38 P_1 + 756 \\
    C_2(P_2) &= 0.1 P_2^2 + 46 P_2 + 451 \\
    C_3(P_3) &= 0.25 P_3^2 + 40 P_3 + 1049
\end{align*}

Sujeito à restrição de balanço de potência (igualdade):

\begin{equation}
    P_1 + P_2 + P_3 = 850 + P_L
\end{equation}

Onde $P_L$ é a perda na transmissão:

\begin{equation}
    P_L = \sum_{i=1}^3 \sum_{j=1}^3 P_i B_{ij} P_j
\end{equation}

Com a matriz de coeficientes de perda $B$:

\begin{equation}
    B = \begin{bmatrix}
    0.000049 & 0.000014 & 0.000015 \\
    0.000014 & 0.000045 & 0.000016 \\
    0.000015 & 0.000016 & 0.000039
    \end{bmatrix}
\end{equation}

E as restrições de capacidade (desigualdades):

\begin{align*}
    150 &\leq P_1 \leq 600 \\
    100 &\leq P_2 \leq 400 \\
    50 &\leq P_3 \leq 200
\end{align*}



\section{Conclusão}

\bibliographystyle{plain}
\bibliography{references}

\end{document}
