\documentclass[12pt,a4paper]{article}

\usepackage[utf8]{inputenc}
\usepackage[T1]{fontenc}
\usepackage[portuguese]{babel}
\usepackage{amsmath,amsfonts,amssymb}
\usepackage{graphicx}
\usepackage[a4paper,margin=2cm]{geometry}
\usepackage{float}
\usepackage{hyperref}
\usepackage{caption}
\usepackage{cite}
\usepackage{listings}
\usepackage{enumitem}
\usepackage{xcolor}

\lstset{
  language=Python,
  basicstyle=\ttfamily\small,
  numbers=left,
  numberstyle=\tiny,
  frame=single,
  breaklines=true,
  keywordstyle=\color{blue}\bfseries,
  stringstyle=\color{red},
  commentstyle=\color{green!60!black}\itshape,
  showstringspaces=false
}

\graphicspath{{./}{../}{../../}{../../plots/}{../../plots/img/}}

\title{
{\footnotesize Universidade Federal de Minas Gerais\\
Escola de Engenharia\\
ELE077 – Otimização Não Linear\\}
\vspace{5mm}
\textbf{Trabalho Computacional II : Otimização Não Linear}
}

\author{
Áquila Oliveira Souza — 2021019327 \and
Josoé Santos Queiroz — 2019026982
}

\date{}

\begin{document}

\maketitle



\section{Introdução}

Problemas de otimização com restrições aparecem em diversas áreas da engenharia, especialmente quando é necessário buscar a melhor solução possível sem violar limites físicos, de segurança ou de operação. Neste trabalho, estudamos três métodos numéricos amplamente utilizados para resolver esse tipo de problema: \textit{Penalidade Interior}, \textit{Penalidade Exterior} e \textit{Lagrangeano Aumentado}. A aplicação prática é feita em dois cenários distintos: o problema estrutural da treliça de três barras e o problema de despacho econômico em sistemas de potência. O objetivo é comparar o desempenho dos métodos, analisar sua convergência e discutir suas vantagens e limitações ao tratar restrições de desigualdade e igualdade.

\section{Revisão da Literatura}

A otimização não linear com restrições é tradicionalmente tratada convertendo-se o problema original em uma sequência de problemas irrestritos, resolvidos com métodos de busca desenvolvidos para otimização sem restrições. Segundo a literatura clássica de otimização \cite{rao2019}, os métodos de penalidade e barreira foram os primeiros a propor essa estratégia, adicionando termos de penalização à função objetivo para forçar o atendimento das restrições.

O \textit{método de Penalidade Exterior} penaliza soluções inviáveis adicionando termos quadráticos que crescem rapidamente quando há violação das restrições. Ele é simples de implementar, porém sofre com mal-condicionamento numérico quando o parâmetro de penalidade se torna muito grande, o que limita a precisão final.

O \textit{método de Penalidade Interior (Barreira)} atua de forma inversa: impede o algoritmo de sair da região viável por meio de funções barreira que divergem ao se aproximar da fronteira das restrições. Esse método aproxima a solução sempre pelo interior do conjunto viável, mas não funciona para restrições de igualdade e também apresenta sensibilidade quando o parâmetro tende a zero.

O \textit{método do Lagrangeano Aumentado} combina a ideia dos multiplicadores de Lagrange com termos de penalidade suavizados, evitando o mal-condicionamento típico dos métodos de penalidade pura. Ele permite convergência estável mesmo com pontos iniciais inviáveis e é considerado o método mais robusto entre os três


\section{Problema 1: Treliça de Três Barras}

O objetivo do problema de otimização restrita é minimizar o gasto de material de uma treliça de três barras. Sendo que $x_1 = A_1$ é a área da seção transversal das barras laterais e $x_2 = A_2$ é a área da seção transversal da barra central.

A função objetivo representa o volume total (ou peso, proporcional ao comprimento) das barras:

\begin{equation}
    \min f(x_1, x_2) = (2\sqrt{2})x_1 + x_2
\end{equation}

As restrições impostas ao problema são baseadas nos limites de estresse e dimensões físicas:

1. Estresse máximo das barras 1 e 2:

\begin{equation}
    P \frac{x_2 + x_1 \sqrt{2}}{x_1^2 \sqrt{2} + 2x_1 x_2} \leq 20
\end{equation}

\begin{equation}
    P \frac{1}{x_1 + x_2 \sqrt{2}} \leq 20
\end{equation}

2. Estresse mínimo para a barra 3:

\begin{equation}
    -P \frac{x_2}{x_1^2 \sqrt{2} + 2x_1 x_2} \leq -5
\end{equation}

3. Limites de tamanho para as barras:

\begin{equation}
    0.1 \leq x_1, x_2 \leq 5
    \label{eq:p1_limits}
\end{equation}

Onde a carga aplicada é $P = 20$.

O espaço de soluções e o ponto inicial sugerido $\mathbf{x} = [1, 3]^T$ foi
plotado na figura, os limites de $x_1$ e $x_2$ na figura são os mesmos da restrição
\eqref{eq:p1_limits}.

\begin{figure}[h!]
\centering
\includegraphics[width=0.6\textwidth]{./img/domain_p1.png}
\caption{Espaço de soluções do problema da treliça de três barras.}
\label{fig:p1_sol_space}
\end{figure}

\subsection{Resultados}

A otimização foi feita usando os métodos de busca restrita de Penalidade Exterior
Penalidade Interior e Lagrangeano Aumentado. 
Inicialmente todos os métodos foram configurados com busca unidimensional por 
Seção Áurea (precisão $10^{-6}$), método do Gradiente como solucionador irrestrito e 
chute inicial $\mathbf{x_0} = [1, 3]^T$. O método das penalidades interiores 
não convergiu para o dado problema, contudo os outros solucionadores sim. As soluções obtidas estão resumidas
na tabela \ref{table:first_results}.



\begin{table}[h]
\centering
\begin{tabular}{lrrrrr}
\hline
 Método                &      $x_1$ &      $x_2$ &   $f(x_1, x_2)$ &   Iterações &   Avaliações \\
\hline
 Penalidade Exterior   & 0.786953 & 0.407355 &       2.63607 &           2 &        13072 \\
 Penalidade Interior   & 1        & 3        &       5.82843 &           0 &         1788 \\
 Lagrangeano Aumentado & 0.787597 & 0.407673 &       2.63804 &           2 &        13178 \\
\hline
\end{tabular}
\caption{Resultados iniciais para $\mathbf{x} = [1, 3]^T$}
\label{table:first_results}
\end{table}


A não convergência do método de penalidade interior é facilmente explicada pela 
natureza do algorítimo: O ponto inicial do problema deve estar dentro da região 
factível e a busca pela solução deve desacelerar ao se aproximar das bordas 
do politopo. Para isso fizemos uma visualização do comportamento do método ao passo
de que ajustamos as penalidades, a desaceleração e a velocidade de aceleração do
método irrestrito. A figura \ref{fig:interior_penalty} mostra o comportamento do método
de penalidade interior para diferentes configurações de penalidade e aceleração.
Foi a partir dessa análise que escolhemos os valores de penalidade e aceleração que
resultaram na melhor aproximação da solução ótima mostrada na tabela \ref{table:interior_results}.

\begin{figure}[h!]
\centering
\includegraphics[width=0.8\textwidth]{./img/comparacao_internas.png}
\caption{Comportamento do método de penalidade interior para diferentes configurações.}
\label{fig:interior_penalty}
\end{figure}

\begin{table}[h]
\centering
\begin{tabular}{lrrrrr}
\hline
 Método                &      $x_1$ &      $x_2$ &   $f(x_1, x_2)$ &   Iterações &   Avaliações \\
\hline
 Penalidade Interior   & 0.77788201 & 0.47797759 &       2.7267 &           8 &        11191 \\
\hline
\end{tabular}
\caption{Resultados finais para $\mathbf{x} = [1, 3]^T$}
\label{table:interior_results}
\end{table}

\begin{table}[h]
\centering
\begin{tabular}{rrr}
\hline
 Aceleração mono-objetivo                &  Penalidade & Desaceleração    \\
\hline
 1.3   & 6 & 0.4  \\
\hline
\end{tabular}
\caption{Parâmetros escolhidos para o método de penalidade interior.}
\label{table:interior_params}
\end{table}

Nesse experimento foi possível observar que o método de penalidade interior é sensível
a diversos ajustes de parâmetros. O que o torna menos robusto que os outros métodos
testados. Além disso a convergência do método foi mais lenta que a dos outros métodos e
sua precisão foi menor. Esse fato pode ser explicado pelo uso de uma função barreira
que se assemelha com uma potência negativa, o que distâcia o mínimo da função objetivo
modificada do mínimo da função original. A figura \ref{fig:final_results} 
mostra a comparação dos ótimos encontrados pelos métodos de penalidade exterior e lagrangeano Aumentado
contra a solução do método de penalidades interiores.

\begin{figure}[h!]
\centering
\includegraphics[width=0.6\textwidth]{./img/comp_resultados.png}
\caption{Comparação dos resultados finais dos métodos testados.}
\label{fig:final_results}    
\end{figure}

\section{Problema 2: Despacho Econômico de Energia}
O problema consiste em determinar a distribuição ótima de geração entre as unidades geradoras disponíveis para atender à demanda de energia elétrica ao menor custo possível.

A função objetivo é minimizar o custo total de geração:

\begin{equation}
    \min f(P_1, P_2, P_3) = C_1(P_1) + C_2(P_2) + C_3(P_3)
\end{equation}

Onde as curvas de custo são dadas por:

\begin{align*}
    C_1(P_1) &= 0.15 P_1^2 + 38 P_1 + 756 \\
    C_2(P_2) &= 0.1 P_2^2 + 46 P_2 + 451 \\
    C_3(P_3) &= 0.25 P_3^2 + 40 P_3 + 1049
\end{align*}

Sujeito à restrição de balanço de potência (igualdade):

\begin{equation}
    P_1 + P_2 + P_3 = 850 + P_L
\end{equation}

Onde $P_L$ é a perda na transmissão:

\begin{equation}
    P_L = \sum_{i=1}^3 \sum_{j=1}^3 P_i B_{ij} P_j
\end{equation}

Com a matriz de coeficientes de perda $B$:

\begin{equation}
    B = \begin{bmatrix}
    0.000049 & 0.000014 & 0.000015 \\
    0.000014 & 0.000045 & 0.000016 \\
    0.000015 & 0.000016 & 0.000039
    \end{bmatrix}
\end{equation}

E as restrições de capacidade (desigualdades):

\begin{align*}
    150 &\leq P_1 \leq 600 \\
    100 &\leq P_2 \leq 400 \\
    50 &\leq P_3 \leq 200
\end{align*}

\subsection{Resultados Numéricos}

Ambos os métodos foram configurados com busca unidimensional por Seção Áurea (precisão $10^{-6}$), método do Gradiente como solucionador irrestrito e chute inicial $x_0 = [600, 400, 200]$ MW, aplicando as mesmas restrições de balanço e de capacidade. As soluções convergiram para valores praticamente idênticos de geração, garantindo o atendimento da carga de 850 MW acrescida das perdas calculadas pela matriz $B$.

\begin{table}[h]
\centering
\begin{tabular}{lccc}
\hline
Método & $P_1$ (MW) & $P_2$ (MW) & $P_3$ (MW) \\
\hline
Penalidade Exterior & 294.8470 & 399.2919 & 175.5602 \\
Lagrangeano Aumentado & 294.9242 & 399.2341 & 175.5714 \\
\hline
\end{tabular}
\end{table}

A Penalidade Exterior atingiu o ponto ótimo em 20 iterações, enquanto o Lagrangeano Aumentado precisou de 19. Apesar da expectativa teórica de maior rapidez do Lagrangeano, a diferença observada foi pequena e pode ser atribuída às escolhas de penalidade e atualização de multiplicadores. Em ambos os casos, a unidade 2 opera próxima ao limite superior (400 MW) por ser a mais econômica, enquanto $P_1$ e $P_3$ ajustam-se para satisfazer o balanço potência–perdas. A convergência para praticamente a mesma solução ocorre devido ao tamanho do problema, pois a principal diferença é que a penalidade externa simplesmente adiciona termos de penalidade à função objetivo, enquanto o Lagrangiano Aumentado combina essa penalidade com os multiplicadores de Lagrange de forma mais sofisticada, incorporando um termo quadrático para acelerar a convergência e melhorar a estabilidade. O método do Lagrangiano Aumentado é uma evolução do método da penalidade, sendo mais robusto em problemas de otimização não linear com restrições.



\section{Conclusão}

Este trabalho explorou métodos clássicos de otimização não linear com restrições. 
A comparação dos métodos comportamentos e soluções obtidas evidenciou as vantagens
e limitações de cada abordagem. Foi possível observar a eficiência do método do
Lagrangeano Aumentado em termos de convergência e robustez, especialmente em
problemas com múltiplas restrições. O método de penalidade exterior mostrou-se simples
e eficaz, embora sensível à escolha dos parâmetros de penalidade. Já o método de
penalidade interior apresentou desafios de convergência, destacando a importância
de um ponto inicial factível e do ajuste cuidadoso dos parâmetros.

\bibliographystyle{plain}
\bibliography{references}

\end{document}
